\documentclass{book}
\usepackage[a4paper,top=2.5cm,bottom=2.5cm,left=2.5cm,right=2.5cm]{geometry}
\usepackage{makeidx}
\usepackage{natbib}
\usepackage{graphicx}
\usepackage{multicol}
\usepackage{float}
\usepackage{listings}
\usepackage{color}
\usepackage{ifthen}
\usepackage[table]{xcolor}
\usepackage{textcomp}
\usepackage{alltt}
\usepackage{ifpdf}
\ifpdf
\usepackage[pdftex,
            pagebackref=true,
            colorlinks=true,
            linkcolor=blue,
            unicode
           ]{hyperref}
\else
\usepackage[ps2pdf,
            pagebackref=true,
            colorlinks=true,
            linkcolor=blue,
            unicode
           ]{hyperref}
\usepackage{pspicture}
\fi
\usepackage[utf8]{inputenc}
\usepackage[french]{babel}

\usepackage{mathptmx}
\usepackage[scaled=.90]{helvet}
\usepackage{courier}
\usepackage{sectsty}
\usepackage{amssymb}
\usepackage[titles]{tocloft}
\usepackage{doxygen}
\lstset{language=C++,inputencoding=utf8,basicstyle=\footnotesize,breaklines=true,breakatwhitespace=true,tabsize=4,numbers=left }
\makeindex
\setcounter{tocdepth}{3}
\renewcommand{\footrulewidth}{0.4pt}
\renewcommand{\familydefault}{\sfdefault}
\hfuzz=15pt
\setlength{\emergencystretch}{15pt}
\hbadness=750
\tolerance=750
\begin{document}
\hypersetup{pageanchor=false,citecolor=blue}
\begin{titlepage}
\vspace*{7cm}
\begin{center}
{\Large Ruzzle Solver }\\
\vspace*{1cm}
{\large Généré par Doxygen 1.8.3.1}\\
\vspace*{0.5cm}
{\small Lundi Novembre 16 2015 09:41:53}\\
\end{center}
\end{titlepage}
\clearemptydoublepage
\pagenumbering{roman}
\tableofcontents
\clearemptydoublepage
\pagenumbering{arabic}
\hypersetup{pageanchor=true,citecolor=blue}
\chapter{Page principale}
\label{index}\hypertarget{index}{}Il faut former le plus de mots possibles avec les seize lettres disponibles dans une grille de quatre par quatre. Les mots doivent être au moins de deux lettres, et trouvés en utilisant des lettres adjacentes les unes aux autres sans réutiliser deux fois la même case de la grille. Les formes conjuguées des verbes sont acceptées.

Comme au Scrabble chaque lettre a une valeur en fonction de sa difficulté à être placée dans un mot et certaines cases sont spéciales \-:


\begin{DoxyItemize}
\item Lettre compte double \-: la lettre dans cette case compte deux fois plus de points
\item Lettre compte triple \-: la lettre dans cette case compte trois fois plus de points
\item Mot compte double \-: les mots contenant cette case comptent deux fois plus de points
\item Mot compte triple \-: les mots contenant cette case comptent trois fois plus de points
\end{DoxyItemize}

Il existe également un bonus en fonction de la longueur du mot créé \-:


\begin{DoxyItemize}
\item 5 lettres \-: 5 points supplémentaires
\item 6 lettres \-: 10 points supplémentaires
\item 7 lettres \-: 15 points supplémentaires
\item 8 lettres \-: 20 points supplémentaires
\item 9 lettres ou plus \-: 25 points supplémentaires
\end{DoxyItemize}

\subsection*{Compilation}

\begin{quotation}
make

\end{quotation}


Permet de compiler l'ensemble des sources, l'exécutable généré peut être retrouvé dans ./bin .

\begin{quotation}
make clean

\end{quotation}


Permet de supprimer tous les .o créé lors de la compilation.

\subsection*{Execution}

\begin{quotation}
./bin/ruzzlesolver

\end{quotation}


Permet d'exécuter le programme ruzzlesolver. 
\chapter{Index des classes}
\section{Liste des classes}
Liste des classes, structures, unions et interfaces avec une brève description \-:\begin{DoxyCompactList}
\item\contentsline{section}{\hyperlink{structelement}{element} \\*Definition de la structure de la pile qui contient des coordonnées x,y, la matrice chemin qui indique le chemin par lequel on est déjà passé, la matrice chemin\-Mot qui contient les lettres du mot dans une matrice vide et le pointeur sur l'élement suivant }{\pageref{structelement}}{}
\item\contentsline{section}{\hyperlink{structt__case}{t\-\_\-case} \\*Definition d'une case qui est une lettre avec son bonus qui lui est associé }{\pageref{structt__case}}{}
\item\contentsline{section}{\hyperlink{structt__score}{t\-\_\-score} \\*Definition d'une structure qui permet de rassembler le mot et son nombre de point }{\pageref{structt__score}}{}
\end{DoxyCompactList}

\chapter{Index des fichiers}
\section{Liste des fichiers}
Liste de tous les fichiers documentés avec une brève description \-:\begin{DoxyCompactList}
\item\contentsline{section}{include/\hyperlink{affichage_8h}{affichage.\-h} \\*Fichier qui contient tout les prototypes de \hyperlink{affichage_8c}{affichage.\-c} }{\pageref{affichage_8h}}{}
\item\contentsline{section}{include/\hyperlink{couleur_8h}{couleur.\-h} \\*Fichier qui contient les prototypes de l'utilisation des couleurs }{\pageref{couleur_8h}}{}
\item\contentsline{section}{include/\hyperlink{generer_8h}{generer.\-h} \\*Fichier qui contient tout les prototypes de \hyperlink{generer_8c}{generer.\-c} }{\pageref{generer_8h}}{}
\item\contentsline{section}{include/\hyperlink{pile__ptr_8h}{pile\-\_\-ptr.\-h} \\*Fichier qui contient tout les prototypes de \hyperlink{pile__ptr_8c}{pile\-\_\-ptr.\-c} }{\pageref{pile__ptr_8h}}{}
\item\contentsline{section}{include/\hyperlink{points_8h}{points.\-h} \\*Fichier qui contient tout les prototypes de \hyperlink{points_8c}{points.\-c} }{\pageref{points_8h}}{}
\item\contentsline{section}{include/\hyperlink{structure_8h}{structure.\-h} \\*Fichier qui contient toutes les structures et énumération }{\pageref{structure_8h}}{}
\item\contentsline{section}{include/\hyperlink{traitement_8h}{traitement.\-h} \\*Fichier qui contient tout les prototypes de \hyperlink{traitement_8c}{traitement.\-c} }{\pageref{traitement_8h}}{}
\item\contentsline{section}{src/\hyperlink{affichage_8c}{affichage.\-c} \\*Fichier qui contient les fonctions qui permettent de réaliser de l'affichage }{\pageref{affichage_8c}}{}
\item\contentsline{section}{src/\hyperlink{generer_8c}{generer.\-c} \\*Fichier qui contient les fonctions qui permettent generer la grille }{\pageref{generer_8c}}{}
\item\contentsline{section}{src/\hyperlink{main_8c}{main.\-c} \\*Fichier qui contient la fonction main et donc les appels des fonctions principales }{\pageref{main_8c}}{}
\item\contentsline{section}{src/\hyperlink{pile__ptr_8c}{pile\-\_\-ptr.\-c} \\*Fichier qui contient les primitives d'accès à la pile }{\pageref{pile__ptr_8c}}{}
\item\contentsline{section}{src/\hyperlink{points_8c}{points.\-c} \\*Fichier qui contient toutes les fonctions qui comptent le nombre de point d'un mot }{\pageref{points_8c}}{}
\item\contentsline{section}{src/\hyperlink{traitement_8c}{traitement.\-c} \\*Fichier qui contient toutes les fonctions qui traitent la grille }{\pageref{traitement_8c}}{}
\end{DoxyCompactList}

\chapter{Documentation des classes}
\hypertarget{structelement}{\section{Référence de la structure element}
\label{structelement}\index{element@{element}}
}


Definition de la structure de la pile qui contient des coordonnées x,y, la matrice chemin qui indique le chemin par lequel on est déjà passé, la matrice chemin\-Mot qui contient les lettres du mot dans une matrice vide et le pointeur sur l'élement suivant.  




{\ttfamily \#include $<$Structure$>$}

\subsection*{Attributs publics}
\begin{DoxyCompactItemize}
\item 
\hypertarget{structelement_a18e88a462664541872b1018c06289ead}{int {\bfseries x}}\label{structelement_a18e88a462664541872b1018c06289ead}

\item 
\hypertarget{structelement_a434c348427ff30755ba723ab1180d070}{int {\bfseries y}}\label{structelement_a434c348427ff30755ba723ab1180d070}

\item 
\hypertarget{structelement_a28553e86dd9c0b009a5abf35a0aabc0c}{int {\bfseries chemin} \mbox{[}N\mbox{]}\mbox{[}N\mbox{]}}\label{structelement_a28553e86dd9c0b009a5abf35a0aabc0c}

\item 
\hypertarget{structelement_a354f5cb4a63d5fac439df62ce3c9772a}{char {\bfseries chemin\-Mot} \mbox{[}N\mbox{]}\mbox{[}N\mbox{]}}\label{structelement_a354f5cb4a63d5fac439df62ce3c9772a}

\item 
\hypertarget{structelement_a4f1534ded1f9373e0843568bf94ea754}{struct \hyperlink{structelement}{element} $\ast$ {\bfseries suivant}}\label{structelement_a4f1534ded1f9373e0843568bf94ea754}

\end{DoxyCompactItemize}


\subsection{Description détaillée}
Definition de la structure de la pile qui contient des coordonnées x,y, la matrice chemin qui indique le chemin par lequel on est déjà passé, la matrice chemin\-Mot qui contient les lettres du mot dans une matrice vide et le pointeur sur l'élement suivant. 

Définition à la ligne 36 du fichier structure.\-h.



La documentation de cette structure a été générée à partir du fichier suivant \-:\begin{DoxyCompactItemize}
\item 
include/\hyperlink{structure_8h}{structure.\-h}\end{DoxyCompactItemize}

\hypertarget{structt__case}{\section{Référence de la structure t\-\_\-case}
\label{structt__case}\index{t\-\_\-case@{t\-\_\-case}}
}


Definition d'une case qui est une lettre avec son bonus qui lui est associé.  




{\ttfamily \#include $<$Structure$>$}

\subsection*{Attributs publics}
\begin{DoxyCompactItemize}
\item 
\hypertarget{structt__case_a38245c7046b2e01295c369509a87ab1b}{int {\bfseries lettre}}\label{structt__case_a38245c7046b2e01295c369509a87ab1b}

\item 
\hypertarget{structt__case_a5648b98d34144d64dfc9ff8397159db5}{enum \hyperlink{structure_8h_a4dfb649c60c07e69175f04af6e69d33e}{t\-\_\-bonus} {\bfseries bonus}}\label{structt__case_a5648b98d34144d64dfc9ff8397159db5}

\end{DoxyCompactItemize}


\subsection{Description détaillée}
Definition d'une case qui est une lettre avec son bonus qui lui est associé. 

Définition à la ligne 24 du fichier structure.\-h.



La documentation de cette structure a été générée à partir du fichier suivant \-:\begin{DoxyCompactItemize}
\item 
include/\hyperlink{structure_8h}{structure.\-h}\end{DoxyCompactItemize}

\hypertarget{structt__score}{\section{Référence de la structure t\-\_\-score}
\label{structt__score}\index{t\-\_\-score@{t\-\_\-score}}
}


Definition d'une structure qui permet de rassembler le mot et son nombre de point.  




{\ttfamily \#include $<$Structure$>$}

\subsection*{Attributs publics}
\begin{DoxyCompactItemize}
\item 
\hypertarget{structt__score_abd797278aa1e44c0c12eaf3ff109807c}{char {\bfseries mot} \mbox{[}20\mbox{]}}\label{structt__score_abd797278aa1e44c0c12eaf3ff109807c}

\item 
\hypertarget{structt__score_afee8a31bb8ce610726a882842f15b29a}{int {\bfseries points}}\label{structt__score_afee8a31bb8ce610726a882842f15b29a}

\end{DoxyCompactItemize}


\subsection{Description détaillée}
Definition d'une structure qui permet de rassembler le mot et son nombre de point. 

Définition à la ligne 30 du fichier structure.\-h.



La documentation de cette structure a été générée à partir du fichier suivant \-:\begin{DoxyCompactItemize}
\item 
include/\hyperlink{structure_8h}{structure.\-h}\end{DoxyCompactItemize}

\chapter{Documentation des fichiers}
\hypertarget{affichage_8h}{}\section{Référence du fichier include/affichage.h}
\label{affichage_8h}\index{include/affichage.\+h@{include/affichage.\+h}}


Fichier qui contient tout les prototypes de \hyperlink{affichage_8c}{affichage.\+c}.  


\subsection*{Macros}
\begin{DoxyCompactItemize}
\item 
\hypertarget{affichage_8h_a0240ac851181b84ac374872dc5434ee4}{}\#define {\bfseries N}~4\label{affichage_8h_a0240ac851181b84ac374872dc5434ee4}

\end{DoxyCompactItemize}
\subsection*{Fonctions}
\begin{DoxyCompactItemize}
\item 
void \hyperlink{affichage_8h_a1d115f06ae228b9b9d49509db742df3b}{afficher\+\_\+matrice} (\hyperlink{structt__case}{t\+\_\+case} grille\mbox{[}N\mbox{]}\mbox{[}N\mbox{]})
\begin{DoxyCompactList}\small\item\em Fonction qui permet d\textquotesingle{}afficher la grille et sa légende. \end{DoxyCompactList}\item 
\hypertarget{affichage_8h_a01924151da096c83254ba825f4480e31}{}void {\bfseries afficher\+\_\+bonus} (\hyperlink{structt__case}{t\+\_\+case} grille\mbox{[}N\mbox{]}\mbox{[}N\mbox{]})\label{affichage_8h_a01924151da096c83254ba825f4480e31}

\item 
void \hyperlink{affichage_8h_ac975ea188aa8a77155922a84cc711366}{afficher\+\_\+liste} (int compteur, \hyperlink{structt__score}{t\+\_\+score} T\mbox{[}1000\mbox{]})
\begin{DoxyCompactList}\small\item\em Fonction qui permet d\textquotesingle{}afficher la liste des mots trié en fonction de leur score. \end{DoxyCompactList}\end{DoxyCompactItemize}


\subsection{Description détaillée}
Fichier qui contient tout les prototypes de \hyperlink{affichage_8c}{affichage.\+c}. 

\begin{DoxyAuthor}{Auteur}
B\+O\+U\+V\+E\+T Rémi \& P\+R\+A\+D\+E\+R\+E-\/\+N\+I\+Q\+U\+E\+T Alexandre 
\end{DoxyAuthor}
\begin{DoxyVersion}{Version}
1.\+0 
\end{DoxyVersion}
\begin{DoxyDate}{Date}
15 novembre 2015 
\end{DoxyDate}


\subsection{Documentation des fonctions}
\hypertarget{affichage_8h_ac975ea188aa8a77155922a84cc711366}{}\index{affichage.\+h@{affichage.\+h}!afficher\+\_\+liste@{afficher\+\_\+liste}}
\index{afficher\+\_\+liste@{afficher\+\_\+liste}!affichage.\+h@{affichage.\+h}}
\subsubsection[{afficher\+\_\+liste(int compteur, t\+\_\+score T[1000])}]{\setlength{\rightskip}{0pt plus 5cm}void afficher\+\_\+liste (
\begin{DoxyParamCaption}
\item[{int}]{nbmot, }
\item[{{\bf t\+\_\+score}}]{T\mbox{[}1000\mbox{]}}
\end{DoxyParamCaption}
)}\label{affichage_8h_ac975ea188aa8a77155922a84cc711366}


Fonction qui permet d\textquotesingle{}afficher la liste des mots trié en fonction de leur score. 


\begin{DoxyParams}{Paramètres}
{\em Prend} & en paramètre le nombre de mot et le tableau des mots et des scores. \\
\hline
\end{DoxyParams}
\begin{DoxyReturn}{Renvoie}
Ne retourne rien. 
\end{DoxyReturn}


Définition à la ligne 75 du fichier affichage.\+c.

\hypertarget{affichage_8h_a1d115f06ae228b9b9d49509db742df3b}{}\index{affichage.\+h@{affichage.\+h}!afficher\+\_\+matrice@{afficher\+\_\+matrice}}
\index{afficher\+\_\+matrice@{afficher\+\_\+matrice}!affichage.\+h@{affichage.\+h}}
\subsubsection[{afficher\+\_\+matrice(t\+\_\+case grille[N][N])}]{\setlength{\rightskip}{0pt plus 5cm}void afficher\+\_\+matrice (
\begin{DoxyParamCaption}
\item[{{\bf t\+\_\+case}}]{grille\mbox{[}\+N\mbox{]}\mbox{[}\+N\mbox{]}}
\end{DoxyParamCaption}
)}\label{affichage_8h_a1d115f06ae228b9b9d49509db742df3b}


Fonction qui permet d\textquotesingle{}afficher la grille et sa légende. 


\begin{DoxyParams}{Paramètres}
{\em Prend} & en paramètre la grille. \\
\hline
\end{DoxyParams}
\begin{DoxyReturn}{Renvoie}
Ne retourne rien. 
\end{DoxyReturn}


Définition à la ligne 20 du fichier affichage.\+c.


\hypertarget{couleur_8h}{\section{Référence du fichier include/couleur.h}
\label{couleur_8h}\index{include/couleur.\-h@{include/couleur.\-h}}
}


Fichier qui contient les prototypes de l'utilisation des couleurs.  


{\ttfamily \#include $<$stdio.\-h$>$}\\*
\subsection*{Macros}
\begin{DoxyCompactItemize}
\item 
\hypertarget{couleur_8h_ada0430ff0133c72d1ab38ca336610f5a}{\#define {\bfseries clrscr}()~printf(\char`\"{}\textbackslash{}033\mbox{[}H\textbackslash{}033\mbox{[}2\-J\char`\"{})}\label{couleur_8h_ada0430ff0133c72d1ab38ca336610f5a}

\item 
\hypertarget{couleur_8h_aabcb2d6536b6c0ab41f99493c911489b}{\#define {\bfseries couleur}(param)~printf(\char`\"{}\textbackslash{}033\mbox{[}\%im\char`\"{},param)}\label{couleur_8h_aabcb2d6536b6c0ab41f99493c911489b}

\end{DoxyCompactItemize}


\subsection{Description détaillée}
Fichier qui contient les prototypes de l'utilisation des couleurs. \begin{DoxyAuthor}{Auteur}
B\-O\-U\-V\-E\-T Rémi \& P\-R\-A\-D\-E\-R\-E-\/\-N\-I\-Q\-U\-E\-T Alexandre 
\end{DoxyAuthor}
\begin{DoxyVersion}{Version}
1.\-0 
\end{DoxyVersion}
\begin{DoxyDate}{Date}
15 novembre 2015 
\end{DoxyDate}


Définition dans le fichier \hyperlink{couleur_8h_source}{couleur.\-h}.


\hypertarget{generer_8h}{\section{Référence du fichier include/generer.h}
\label{generer_8h}\index{include/generer.\-h@{include/generer.\-h}}
}


Fichier qui contient tout les prototypes de \hyperlink{generer_8c}{generer.\-c}.  


\subsection*{Macros}
\begin{DoxyCompactItemize}
\item 
\hypertarget{generer_8h_a0240ac851181b84ac374872dc5434ee4}{\#define {\bfseries N}~4}\label{generer_8h_a0240ac851181b84ac374872dc5434ee4}

\end{DoxyCompactItemize}
\subsection*{Fonctions}
\begin{DoxyCompactItemize}
\item 
int \hyperlink{generer_8h_a31044302280c0ec9eca89286f81127af}{rand\-\_\-a\-\_\-b} (int a, int b)
\begin{DoxyCompactList}\small\item\em Fonction qui permet de générer un nombre aléatoire. \end{DoxyCompactList}\item 
void \hyperlink{generer_8h_a7ce74e6424ae2e628d8fea22f43beeb2}{generation} (\hyperlink{structt__case}{t\-\_\-case} grille\mbox{[}N\mbox{]}\mbox{[}N\mbox{]})
\begin{DoxyCompactList}\small\item\em Fonction qui permet de générer la grille de caractère ainsi que les bonus qui lui son associé. \end{DoxyCompactList}\end{DoxyCompactItemize}


\subsection{Description détaillée}
Fichier qui contient tout les prototypes de \hyperlink{generer_8c}{generer.\-c}. \begin{DoxyAuthor}{Auteur}
B\-O\-U\-V\-E\-T Rémi \& P\-R\-A\-D\-E\-R\-E-\/\-N\-I\-Q\-U\-E\-T Alexandre 
\end{DoxyAuthor}
\begin{DoxyVersion}{Version}
1.\-0 
\end{DoxyVersion}
\begin{DoxyDate}{Date}
15 novembre 2015 
\end{DoxyDate}


Définition dans le fichier \hyperlink{generer_8h_source}{generer.\-h}.



\subsection{Documentation des fonctions}
\hypertarget{generer_8h_a7ce74e6424ae2e628d8fea22f43beeb2}{\index{generer.\-h@{generer.\-h}!generation@{generation}}
\index{generation@{generation}!generer.h@{generer.\-h}}
\subsubsection[{generation}]{\setlength{\rightskip}{0pt plus 5cm}void generation (
\begin{DoxyParamCaption}
\item[{{\bf t\-\_\-case}}]{grille\mbox{[}\-N\mbox{]}\mbox{[}\-N\mbox{]}}
\end{DoxyParamCaption}
)}}\label{generer_8h_a7ce74e6424ae2e628d8fea22f43beeb2}


Fonction qui permet de générer la grille de caractère ainsi que les bonus qui lui son associé. 


\begin{DoxyParams}{Paramètres}
{\em Prend} & en paramètre la grille. \\
\hline
\end{DoxyParams}
\begin{DoxyReturn}{Renvoie}
Ne retourne rien. 
\end{DoxyReturn}


Définition à la ligne 33 du fichier generer.\-c.

\hypertarget{generer_8h_a31044302280c0ec9eca89286f81127af}{\index{generer.\-h@{generer.\-h}!rand\-\_\-a\-\_\-b@{rand\-\_\-a\-\_\-b}}
\index{rand\-\_\-a\-\_\-b@{rand\-\_\-a\-\_\-b}!generer.h@{generer.\-h}}
\subsubsection[{rand\-\_\-a\-\_\-b}]{\setlength{\rightskip}{0pt plus 5cm}int rand\-\_\-a\-\_\-b (
\begin{DoxyParamCaption}
\item[{int}]{a, }
\item[{int}]{b}
\end{DoxyParamCaption}
)}}\label{generer_8h_a31044302280c0ec9eca89286f81127af}


Fonction qui permet de générer un nombre aléatoire. 


\begin{DoxyParams}{Paramètres}
{\em Prend} & en paramètre deux entier qui permettent de définir les bornes du nombre aléatoire généré avec \mbox{[}a;b\mbox{[} . \\
\hline
\end{DoxyParams}
\begin{DoxyReturn}{Renvoie}
Retourne le nombre aléatoire. 
\end{DoxyReturn}


Définition à la ligne 22 du fichier generer.\-c.


\hypertarget{pile__ptr_8h}{}\section{Référence du fichier include/pile\+\_\+ptr.h}
\label{pile__ptr_8h}\index{include/pile\+\_\+ptr.\+h@{include/pile\+\_\+ptr.\+h}}


Fichier qui contient tout les prototypes de \hyperlink{pile__ptr_8c}{pile\+\_\+ptr.\+c}.  


\subsection*{Macros}
\begin{DoxyCompactItemize}
\item 
\hypertarget{pile__ptr_8h_a0240ac851181b84ac374872dc5434ee4}{}\#define {\bfseries N}~4\label{pile__ptr_8h_a0240ac851181b84ac374872dc5434ee4}

\end{DoxyCompactItemize}
\subsection*{Fonctions}
\begin{DoxyCompactItemize}
\item 
void \hyperlink{pile__ptr_8h_ac4e8451a9141ddd60fd45e98cf741aad}{initpile} (void)
\begin{DoxyCompactList}\small\item\em Fonction qui initialise la pile. \end{DoxyCompactList}\item 
void \hyperlink{pile__ptr_8h_a7024e12f28fa12e7835b8940054ed4fa}{Copier\+Matrice} (char T1\mbox{[}N\mbox{]}\mbox{[}N\mbox{]}, char T2\mbox{[}N\mbox{]}\mbox{[}N\mbox{]})
\begin{DoxyCompactList}\small\item\em Fonction qui copie une matrice de caractère dans une autre. \end{DoxyCompactList}\item 
void \hyperlink{pile__ptr_8h_afd2bb36dec2c145c8f3a581ba51819ed}{empiler} (int x2, int y2, char chemin2\mbox{[}N\mbox{]}\mbox{[}N\mbox{]})
\begin{DoxyCompactList}\small\item\em Fonction qui ajoute un élément à notre pile. \end{DoxyCompactList}\item 
void \hyperlink{pile__ptr_8h_a2e632b78b736358faa9b599fd08e54e5}{depiler} (int $\ast$x2, int $\ast$y2, char chemin2\mbox{[}N\mbox{]}\mbox{[}N\mbox{]})
\begin{DoxyCompactList}\small\item\em Fonction qui permet de retirer un élément de la pile et d\textquotesingle{}attribuer sa valeur aux élément placés en paramètre. \end{DoxyCompactList}\end{DoxyCompactItemize}


\subsection{Description détaillée}
Fichier qui contient tout les prototypes de \hyperlink{pile__ptr_8c}{pile\+\_\+ptr.\+c}. 

\begin{DoxyAuthor}{Auteur}
B\+O\+U\+V\+E\+T Rémi \& P\+R\+A\+D\+E\+R\+E-\/\+N\+I\+Q\+U\+E\+T Alexandre 
\end{DoxyAuthor}
\begin{DoxyVersion}{Version}
1.\+0 
\end{DoxyVersion}
\begin{DoxyDate}{Date}
15 novembre 2015 
\end{DoxyDate}


\subsection{Documentation des fonctions}
\hypertarget{pile__ptr_8h_a7024e12f28fa12e7835b8940054ed4fa}{}\index{pile\+\_\+ptr.\+h@{pile\+\_\+ptr.\+h}!Copier\+Matrice@{Copier\+Matrice}}
\index{Copier\+Matrice@{Copier\+Matrice}!pile\+\_\+ptr.\+h@{pile\+\_\+ptr.\+h}}
\subsubsection[{Copier\+Matrice(char T1[N][N], char T2[N][N])}]{\setlength{\rightskip}{0pt plus 5cm}void Copier\+Matrice (
\begin{DoxyParamCaption}
\item[{char}]{T1\mbox{[}\+N\mbox{]}\mbox{[}\+N\mbox{]}, }
\item[{char}]{T2\mbox{[}\+N\mbox{]}\mbox{[}\+N\mbox{]}}
\end{DoxyParamCaption}
)}\label{pile__ptr_8h_a7024e12f28fa12e7835b8940054ed4fa}


Fonction qui copie une matrice de caractère dans une autre. 


\begin{DoxyParams}{Paramètres}
{\em Il} & y a deux mattrice de caractère en paramètre \+: T2 qui se copie dans T1. \\
\hline
\end{DoxyParams}
\begin{DoxyReturn}{Renvoie}
Ne retourne rien. 
\end{DoxyReturn}


Définition à la ligne 37 du fichier pile\+\_\+ptr.\+c.

\hypertarget{pile__ptr_8h_a2e632b78b736358faa9b599fd08e54e5}{}\index{pile\+\_\+ptr.\+h@{pile\+\_\+ptr.\+h}!depiler@{depiler}}
\index{depiler@{depiler}!pile\+\_\+ptr.\+h@{pile\+\_\+ptr.\+h}}
\subsubsection[{depiler(int $\ast$x2, int $\ast$y2, char chemin2[N][N])}]{\setlength{\rightskip}{0pt plus 5cm}void depiler (
\begin{DoxyParamCaption}
\item[{int $\ast$}]{x2, }
\item[{int $\ast$}]{y2, }
\item[{char}]{chemin2\mbox{[}\+N\mbox{]}\mbox{[}\+N\mbox{]}}
\end{DoxyParamCaption}
)}\label{pile__ptr_8h_a2e632b78b736358faa9b599fd08e54e5}


Fonction qui permet de retirer un élément de la pile et d\textquotesingle{}attribuer sa valeur aux élément placés en paramètre. 


\begin{DoxyParams}{Paramètres}
{\em Elle} & prend en paramètre les pointeurs $\ast$x2, $\ast$y2 et chemin2. \\
\hline
\end{DoxyParams}
\begin{DoxyReturn}{Renvoie}
Ne retourne rien. 
\end{DoxyReturn}


Définition à la ligne 72 du fichier pile\+\_\+ptr.\+c.

\hypertarget{pile__ptr_8h_afd2bb36dec2c145c8f3a581ba51819ed}{}\index{pile\+\_\+ptr.\+h@{pile\+\_\+ptr.\+h}!empiler@{empiler}}
\index{empiler@{empiler}!pile\+\_\+ptr.\+h@{pile\+\_\+ptr.\+h}}
\subsubsection[{empiler(int x2, int y2, char chemin2[N][N])}]{\setlength{\rightskip}{0pt plus 5cm}void empiler (
\begin{DoxyParamCaption}
\item[{int}]{x2, }
\item[{int}]{y2, }
\item[{char}]{chemin2\mbox{[}\+N\mbox{]}\mbox{[}\+N\mbox{]}}
\end{DoxyParamCaption}
)}\label{pile__ptr_8h_afd2bb36dec2c145c8f3a581ba51819ed}


Fonction qui ajoute un élément à notre pile. 


\begin{DoxyParams}{Paramètres}
{\em Prend} & en paramètre deux coordonnées x2 et y2 et également la matrice de caractère chemin2. \\
\hline
\end{DoxyParams}
\begin{DoxyReturn}{Renvoie}
Ne retourne rien. 
\end{DoxyReturn}


Définition à la ligne 54 du fichier pile\+\_\+ptr.\+c.

\hypertarget{pile__ptr_8h_ac4e8451a9141ddd60fd45e98cf741aad}{}\index{pile\+\_\+ptr.\+h@{pile\+\_\+ptr.\+h}!initpile@{initpile}}
\index{initpile@{initpile}!pile\+\_\+ptr.\+h@{pile\+\_\+ptr.\+h}}
\subsubsection[{initpile(void)}]{\setlength{\rightskip}{0pt plus 5cm}void initpile (
\begin{DoxyParamCaption}
\item[{void}]{}
\end{DoxyParamCaption}
)}\label{pile__ptr_8h_ac4e8451a9141ddd60fd45e98cf741aad}


Fonction qui initialise la pile. 


\begin{DoxyParams}{Paramètres}
{\em Aucun} & paramètre. \\
\hline
\end{DoxyParams}
\begin{DoxyReturn}{Renvoie}
Ne retourne rien. 
\end{DoxyReturn}


Définition à la ligne 26 du fichier pile\+\_\+ptr.\+c.


\hypertarget{points_8h}{}\section{Référence du fichier include/points.h}
\label{points_8h}\index{include/points.\+h@{include/points.\+h}}


Fichier qui contient tout les prototypes de \hyperlink{points_8c}{points.\+c}.  


\subsection*{Macros}
\begin{DoxyCompactItemize}
\item 
\hypertarget{points_8h_a0240ac851181b84ac374872dc5434ee4}{}\#define {\bfseries N}~4\label{points_8h_a0240ac851181b84ac374872dc5434ee4}

\end{DoxyCompactItemize}
\subsection*{Fonctions}
\begin{DoxyCompactItemize}
\item 
int \hyperlink{points_8h_a09c563917508ec5f7b30f5e51a2f41bd}{points\+\_\+lettre} (char lettre)
\begin{DoxyCompactList}\small\item\em Fonction qui permet de connaitre le nombre de point initial d\textquotesingle{}une lettre. \end{DoxyCompactList}\item 
int \hyperlink{points_8h_aa91baeedbdca68768f85f12d05a73214}{points\+\_\+lettre\+\_\+bonus} (\hyperlink{structt__case}{t\+\_\+case} grille\mbox{[}N\mbox{]}\mbox{[}N\mbox{]}, char chemin\mbox{[}N\mbox{]}\mbox{[}N\mbox{]})
\begin{DoxyCompactList}\small\item\em Fonction qui permet de calculer le nombre de point de chaque lettre et de leur bonus associé. \end{DoxyCompactList}\item 
int \hyperlink{points_8h_a26cbb9e7dd284b9268a40830c80cd1bf}{points\+\_\+longueur} (char mot\mbox{[}40\mbox{]})
\begin{DoxyCompactList}\small\item\em Fonction qui permet de calculer nombre de points associé à la longueur du mot. \end{DoxyCompactList}\item 
void \hyperlink{points_8h_af7b76ac111b0aec7744926a43b9f3502}{points\+\_\+mot\+\_\+bonus} (int $\ast$nb\+Points, \hyperlink{structt__case}{t\+\_\+case} grille\mbox{[}N\mbox{]}\mbox{[}N\mbox{]}, char chemin\mbox{[}N\mbox{]}\mbox{[}N\mbox{]})
\begin{DoxyCompactList}\small\item\em Fonction qui permet de calculer le bonus au niveau du mot. \end{DoxyCompactList}\item 
int \hyperlink{points_8h_a2eb22ab537e59f75c27ce523ae3a52e0}{calcul\+Point} (\hyperlink{structt__case}{t\+\_\+case} grille\mbox{[}N\mbox{]}\mbox{[}N\mbox{]}, char mot\mbox{[}40\mbox{]}, char chemin\mbox{[}N\mbox{]}\mbox{[}N\mbox{]})
\begin{DoxyCompactList}\small\item\em Fonction qui permet de calculer le nombre total de point d\textquotesingle{}un mot en prenant en compte sa longueur et ses bonus. \end{DoxyCompactList}\end{DoxyCompactItemize}


\subsection{Description détaillée}
Fichier qui contient tout les prototypes de \hyperlink{points_8c}{points.\+c}. 

\begin{DoxyAuthor}{Auteur}
B\+O\+U\+V\+E\+T Rémi \& P\+R\+A\+D\+E\+R\+E-\/\+N\+I\+Q\+U\+E\+T Alexandre 
\end{DoxyAuthor}
\begin{DoxyVersion}{Version}
1.\+0 
\end{DoxyVersion}
\begin{DoxyDate}{Date}
15 novembre 2015 
\end{DoxyDate}


\subsection{Documentation des fonctions}
\hypertarget{points_8h_a2eb22ab537e59f75c27ce523ae3a52e0}{}\index{points.\+h@{points.\+h}!calcul\+Point@{calcul\+Point}}
\index{calcul\+Point@{calcul\+Point}!points.\+h@{points.\+h}}
\subsubsection[{calcul\+Point(t\+\_\+case grille[N][N], char mot[40], char chemin[N][N])}]{\setlength{\rightskip}{0pt plus 5cm}int calcul\+Point (
\begin{DoxyParamCaption}
\item[{{\bf t\+\_\+case}}]{grille\mbox{[}\+N\mbox{]}\mbox{[}\+N\mbox{]}, }
\item[{char}]{mot\mbox{[}40\mbox{]}, }
\item[{char}]{chemin\mbox{[}\+N\mbox{]}\mbox{[}\+N\mbox{]}}
\end{DoxyParamCaption}
)}\label{points_8h_a2eb22ab537e59f75c27ce523ae3a52e0}


Fonction qui permet de calculer le nombre total de point d\textquotesingle{}un mot en prenant en compte sa longueur et ses bonus. 


\begin{DoxyParams}{Paramètres}
{\em Prend} & en paramètre la grille du jeu, le mot présent dans la grille et son chemin associé. \\
\hline
\end{DoxyParams}
\begin{DoxyReturn}{Renvoie}
Retourne le nombre de point associé au mot. 
\end{DoxyReturn}


Définition à la ligne 133 du fichier points.\+c.

\hypertarget{points_8h_a09c563917508ec5f7b30f5e51a2f41bd}{}\index{points.\+h@{points.\+h}!points\+\_\+lettre@{points\+\_\+lettre}}
\index{points\+\_\+lettre@{points\+\_\+lettre}!points.\+h@{points.\+h}}
\subsubsection[{points\+\_\+lettre(char lettre)}]{\setlength{\rightskip}{0pt plus 5cm}int points\+\_\+lettre (
\begin{DoxyParamCaption}
\item[{char}]{lettre}
\end{DoxyParamCaption}
)}\label{points_8h_a09c563917508ec5f7b30f5e51a2f41bd}


Fonction qui permet de connaitre le nombre de point initial d\textquotesingle{}une lettre. 


\begin{DoxyParams}{Paramètres}
{\em Prend} & un caractère en paramètre. \\
\hline
\end{DoxyParams}
\begin{DoxyReturn}{Renvoie}
Retourne le nombre de point associé à la lettre. 
\end{DoxyReturn}


Définition à la ligne 21 du fichier points.\+c.

\hypertarget{points_8h_aa91baeedbdca68768f85f12d05a73214}{}\index{points.\+h@{points.\+h}!points\+\_\+lettre\+\_\+bonus@{points\+\_\+lettre\+\_\+bonus}}
\index{points\+\_\+lettre\+\_\+bonus@{points\+\_\+lettre\+\_\+bonus}!points.\+h@{points.\+h}}
\subsubsection[{points\+\_\+lettre\+\_\+bonus(t\+\_\+case grille[N][N], char chemin[N][N])}]{\setlength{\rightskip}{0pt plus 5cm}int points\+\_\+lettre\+\_\+bonus (
\begin{DoxyParamCaption}
\item[{{\bf t\+\_\+case}}]{grille\mbox{[}\+N\mbox{]}\mbox{[}\+N\mbox{]}, }
\item[{char}]{chemin\mbox{[}\+N\mbox{]}\mbox{[}\+N\mbox{]}}
\end{DoxyParamCaption}
)}\label{points_8h_aa91baeedbdca68768f85f12d05a73214}


Fonction qui permet de calculer le nombre de point de chaque lettre et de leur bonus associé. 


\begin{DoxyParams}{Paramètres}
{\em Prend} & en paramètre la grille et le chemin associé. \\
\hline
\end{DoxyParams}
\begin{DoxyReturn}{Renvoie}
Retourne le nombre de point associé aux lettres du mot. 
\end{DoxyReturn}


Définition à la ligne 59 du fichier points.\+c.

\hypertarget{points_8h_a26cbb9e7dd284b9268a40830c80cd1bf}{}\index{points.\+h@{points.\+h}!points\+\_\+longueur@{points\+\_\+longueur}}
\index{points\+\_\+longueur@{points\+\_\+longueur}!points.\+h@{points.\+h}}
\subsubsection[{points\+\_\+longueur(char mot[40])}]{\setlength{\rightskip}{0pt plus 5cm}int points\+\_\+longueur (
\begin{DoxyParamCaption}
\item[{char}]{mot\mbox{[}40\mbox{]}}
\end{DoxyParamCaption}
)}\label{points_8h_a26cbb9e7dd284b9268a40830c80cd1bf}


Fonction qui permet de calculer nombre de points associé à la longueur du mot. 


\begin{DoxyParams}{Paramètres}
{\em Prend} & en paramètre un mot. \\
\hline
\end{DoxyParams}
\begin{DoxyReturn}{Renvoie}
Retourne le nombre de point de la longueur du mot. 
\end{DoxyReturn}


Définition à la ligne 85 du fichier points.\+c.

\hypertarget{points_8h_af7b76ac111b0aec7744926a43b9f3502}{}\index{points.\+h@{points.\+h}!points\+\_\+mot\+\_\+bonus@{points\+\_\+mot\+\_\+bonus}}
\index{points\+\_\+mot\+\_\+bonus@{points\+\_\+mot\+\_\+bonus}!points.\+h@{points.\+h}}
\subsubsection[{points\+\_\+mot\+\_\+bonus(int $\ast$nb\+Points, t\+\_\+case grille[N][N], char chemin[N][N])}]{\setlength{\rightskip}{0pt plus 5cm}void points\+\_\+mot\+\_\+bonus (
\begin{DoxyParamCaption}
\item[{int $\ast$}]{nb\+Points, }
\item[{{\bf t\+\_\+case}}]{grille\mbox{[}\+N\mbox{]}\mbox{[}\+N\mbox{]}, }
\item[{char}]{chemin\mbox{[}\+N\mbox{]}\mbox{[}\+N\mbox{]}}
\end{DoxyParamCaption}
)}\label{points_8h_af7b76ac111b0aec7744926a43b9f3502}


Fonction qui permet de calculer le bonus au niveau du mot. 


\begin{DoxyParams}{Paramètres}
{\em Prend} & en paramètre la grille du jeu, le chemin associé et un pointeur sur une varible points. \\
\hline
\end{DoxyParams}
\begin{DoxyReturn}{Renvoie}
Ne retourne rien. 
\end{DoxyReturn}


Définition à la ligne 113 du fichier points.\+c.


\hypertarget{structure_8h}{\section{Référence du fichier include/structure.h}
\label{structure_8h}\index{include/structure.\-h@{include/structure.\-h}}
}


Fichier qui contient toutes les structures et énumération.  


\subsection*{Classes}
\begin{DoxyCompactItemize}
\item 
struct \hyperlink{structt__case}{t\-\_\-case}
\begin{DoxyCompactList}\small\item\em Definition d'une case qui est une lettre avec son bonus qui lui est associé. \end{DoxyCompactList}\item 
struct \hyperlink{structt__score}{t\-\_\-score}
\begin{DoxyCompactList}\small\item\em Definition d'une structure qui permet de rassembler le mot et son nombre de point. \end{DoxyCompactList}\item 
struct \hyperlink{structelement}{element}
\begin{DoxyCompactList}\small\item\em Definition de la structure de la pile qui contient des coordonnées x,y, la matrice chemin qui indique le chemin par lequel on est déjà passé, la matrice chemin\-Mot qui contient les lettres du mot dans une matrice vide et le pointeur sur l'élement suivant. \end{DoxyCompactList}\end{DoxyCompactItemize}
\subsection*{Macros}
\begin{DoxyCompactItemize}
\item 
\hypertarget{structure_8h_a0240ac851181b84ac374872dc5434ee4}{\#define {\bfseries N}~4}\label{structure_8h_a0240ac851181b84ac374872dc5434ee4}

\end{DoxyCompactItemize}
\subsection*{Définitions de type}
\begin{DoxyCompactItemize}
\item 
\hypertarget{structure_8h_a8968705d25c62eaf27310fae6cc2603f}{typedef struct \hyperlink{structelement}{element} {\bfseries t\-\_\-element}}\label{structure_8h_a8968705d25c62eaf27310fae6cc2603f}

\end{DoxyCompactItemize}
\subsection*{Énumérations}
\begin{DoxyCompactItemize}
\item 
enum \hyperlink{structure_8h_a4dfb649c60c07e69175f04af6e69d33e}{t\-\_\-bonus} \{ \\*
{\bfseries aucun}, 
{\bfseries Lettre\-Double}, 
{\bfseries Lettre\-Triple}, 
{\bfseries Mot\-Double}, 
\\*
{\bfseries Mot\-Triple}
 \}
\begin{DoxyCompactList}\small\item\em Cette enumération permet de définir plus facilement les bonus. \end{DoxyCompactList}\end{DoxyCompactItemize}


\subsection{Description détaillée}
Fichier qui contient toutes les structures et énumération. \begin{DoxyAuthor}{Auteur}
B\-O\-U\-V\-E\-T Rémi \& P\-R\-A\-D\-E\-R\-E-\/\-N\-I\-Q\-U\-E\-T Alexandre 
\end{DoxyAuthor}
\begin{DoxyVersion}{Version}
1.\-0 
\end{DoxyVersion}
\begin{DoxyDate}{Date}
15 novembre 2015 
\end{DoxyDate}


Définition dans le fichier \hyperlink{structure_8h_source}{structure.\-h}.


\hypertarget{traitement_8h}{}\section{Référence du fichier include/traitement.h}
\label{traitement_8h}\index{include/traitement.\+h@{include/traitement.\+h}}


Fichier qui contient tout les prototypes de \hyperlink{traitement_8c}{traitement.\+c}.  


\subsection*{Macros}
\begin{DoxyCompactItemize}
\item 
\hypertarget{traitement_8h_a0240ac851181b84ac374872dc5434ee4}{}\#define {\bfseries N}~4\label{traitement_8h_a0240ac851181b84ac374872dc5434ee4}

\end{DoxyCompactItemize}
\subsection*{Fonctions}
\begin{DoxyCompactItemize}
\item 
int \hyperlink{traitement_8h_ae3067c7cd3382c115c5a497ad3f8bd57}{trouverchemin} (\hyperlink{structt__case}{t\+\_\+case} grille\mbox{[}N\mbox{]}\mbox{[}N\mbox{]}, char motdico\mbox{[}40\mbox{]}, int i, int j, char chemin\mbox{[}N\mbox{]}\mbox{[}N\mbox{]})
\begin{DoxyCompactList}\small\item\em Fonction qui permet de parcourir tous les chemins de la matrice à partir d\textquotesingle{}une case et qui calcul le chemin associé si le mot est présent. \end{DoxyCompactList}\item 
int \hyperlink{traitement_8h_a8df66ad8e9a6e13c18b7001561db4194}{motpresent} (\hyperlink{structt__case}{t\+\_\+case} grille\mbox{[}N\mbox{]}\mbox{[}N\mbox{]}, char motdico\mbox{[}40\mbox{]}, char chemin\mbox{[}N\mbox{]}\mbox{[}N\mbox{]})
\begin{DoxyCompactList}\small\item\em Fonction qui permet de determiner si un mot est présent dans la matrice et de calculer son chemin. \end{DoxyCompactList}\item 
void \hyperlink{traitement_8h_af38304f3718d12ae5f3424a12a841103}{trouver\+Liste} (\hyperlink{structt__case}{t\+\_\+case} grille\mbox{[}N\mbox{]}\mbox{[}N\mbox{]})
\begin{DoxyCompactList}\small\item\em Fonction qui enregistre dans un fichier temporaire tous les mots présents dans la matrice ainsi que leurs points associés. \end{DoxyCompactList}\item 
void \hyperlink{traitement_8h_ab160ae699802dc5e10cda59e2d335e73}{tri} (\hyperlink{structt__score}{t\+\_\+score} T\mbox{[}1000\mbox{]}, int $\ast$taille\+Liste)
\begin{DoxyCompactList}\small\item\em Fonction qui récupère tous les mots et leurs points correspondant pour les trier. \end{DoxyCompactList}\end{DoxyCompactItemize}


\subsection{Description détaillée}
Fichier qui contient tout les prototypes de \hyperlink{traitement_8c}{traitement.\+c}. 

\begin{DoxyAuthor}{Auteur}
B\+O\+U\+V\+E\+T Rémi \& P\+R\+A\+D\+E\+R\+E-\/\+N\+I\+Q\+U\+E\+T Alexandre 
\end{DoxyAuthor}
\begin{DoxyVersion}{Version}
1.\+0 
\end{DoxyVersion}
\begin{DoxyDate}{Date}
15 novembre 2015 
\end{DoxyDate}


\subsection{Documentation des fonctions}
\hypertarget{traitement_8h_a8df66ad8e9a6e13c18b7001561db4194}{}\index{traitement.\+h@{traitement.\+h}!motpresent@{motpresent}}
\index{motpresent@{motpresent}!traitement.\+h@{traitement.\+h}}
\subsubsection[{motpresent(t\+\_\+case grille[N][N], char motdico[40], char chemin[N][N])}]{\setlength{\rightskip}{0pt plus 5cm}int motpresent (
\begin{DoxyParamCaption}
\item[{{\bf t\+\_\+case}}]{grille\mbox{[}\+N\mbox{]}\mbox{[}\+N\mbox{]}, }
\item[{char}]{motdico\mbox{[}40\mbox{]}, }
\item[{char}]{chemin\mbox{[}\+N\mbox{]}\mbox{[}\+N\mbox{]}}
\end{DoxyParamCaption}
)}\label{traitement_8h_a8df66ad8e9a6e13c18b7001561db4194}


Fonction qui permet de determiner si un mot est présent dans la matrice et de calculer son chemin. 


\begin{DoxyParams}{Paramètres}
{\em Prend} & en paramètre la grille, le mot à trouver et la matrice chemin. \\
\hline
\end{DoxyParams}
\begin{DoxyReturn}{Renvoie}
Retourne 1 si le mot est présent sinon retourne 0. 
\end{DoxyReturn}


Définition à la ligne 125 du fichier traitement.\+c.

\hypertarget{traitement_8h_ab160ae699802dc5e10cda59e2d335e73}{}\index{traitement.\+h@{traitement.\+h}!tri@{tri}}
\index{tri@{tri}!traitement.\+h@{traitement.\+h}}
\subsubsection[{tri(t\+\_\+score T[1000], int $\ast$taille\+Liste)}]{\setlength{\rightskip}{0pt plus 5cm}void tri (
\begin{DoxyParamCaption}
\item[{{\bf t\+\_\+score}}]{T\mbox{[}1000\mbox{]}, }
\item[{int $\ast$}]{taille\+Liste}
\end{DoxyParamCaption}
)}\label{traitement_8h_ab160ae699802dc5e10cda59e2d335e73}


Fonction qui récupère tous les mots et leurs points correspondant pour les trier. 

void \hyperlink{traitement_8c_ab160ae699802dc5e10cda59e2d335e73}{tri(t\+\_\+score T\mbox{[}1000\mbox{]}, int $\ast$ taille\+Liste)} 
\begin{DoxyParams}{Paramètres}
{\em Prend} & en paramètre le tableau qui stocke les mots et leurs points associés ainsi que un pointeur sur le nombre de mot. \\
\hline
\end{DoxyParams}
\begin{DoxyReturn}{Renvoie}
Ne retourne rien. 
\end{DoxyReturn}


Définition à la ligne 175 du fichier traitement.\+c.

\hypertarget{traitement_8h_ae3067c7cd3382c115c5a497ad3f8bd57}{}\index{traitement.\+h@{traitement.\+h}!trouverchemin@{trouverchemin}}
\index{trouverchemin@{trouverchemin}!traitement.\+h@{traitement.\+h}}
\subsubsection[{trouverchemin(t\+\_\+case grille[N][N], char motdico[40], int i, int j, char chemin[N][N])}]{\setlength{\rightskip}{0pt plus 5cm}int trouverchemin (
\begin{DoxyParamCaption}
\item[{{\bf t\+\_\+case}}]{grille\mbox{[}\+N\mbox{]}\mbox{[}\+N\mbox{]}, }
\item[{char}]{motdico\mbox{[}40\mbox{]}, }
\item[{int}]{i, }
\item[{int}]{j, }
\item[{char}]{chemin\mbox{[}\+N\mbox{]}\mbox{[}\+N\mbox{]}}
\end{DoxyParamCaption}
)}\label{traitement_8h_ae3067c7cd3382c115c5a497ad3f8bd57}


Fonction qui permet de parcourir tous les chemins de la matrice à partir d\textquotesingle{}une case et qui calcul le chemin associé si le mot est présent. 


\begin{DoxyParams}{Paramètres}
{\em Prend} & en paramètre la grille, le mot à trouver, les coordonnées i,j à partir desquels il faut trouver un chemin et la matrice chemin. \\
\hline
\end{DoxyParams}
\begin{DoxyReturn}{Renvoie}
Retourne 1 si le mot est présent sinon retourne 0. 
\end{DoxyReturn}


Définition à la ligne 23 du fichier traitement.\+c.

\hypertarget{traitement_8h_af38304f3718d12ae5f3424a12a841103}{}\index{traitement.\+h@{traitement.\+h}!trouver\+Liste@{trouver\+Liste}}
\index{trouver\+Liste@{trouver\+Liste}!traitement.\+h@{traitement.\+h}}
\subsubsection[{trouver\+Liste(t\+\_\+case grille[N][N])}]{\setlength{\rightskip}{0pt plus 5cm}void trouver\+Liste (
\begin{DoxyParamCaption}
\item[{{\bf t\+\_\+case}}]{grille\mbox{[}\+N\mbox{]}\mbox{[}\+N\mbox{]}}
\end{DoxyParamCaption}
)}\label{traitement_8h_af38304f3718d12ae5f3424a12a841103}


Fonction qui enregistre dans un fichier temporaire tous les mots présents dans la matrice ainsi que leurs points associés. 


\begin{DoxyParams}{Paramètres}
{\em Prend} & en paramètre la grille. \\
\hline
\end{DoxyParams}
\begin{DoxyReturn}{Renvoie}
Ne retourne rien. 
\end{DoxyReturn}


Définition à la ligne 146 du fichier traitement.\+c.


\hypertarget{affichage_8c}{}\section{Référence du fichier src/affichage.c}
\label{affichage_8c}\index{src/affichage.\+c@{src/affichage.\+c}}


Fichier qui contient les fonctions qui permettent de réaliser de l\textquotesingle{}affichage.  


{\ttfamily \#include $<$stdio.\+h$>$}\\*
{\ttfamily \#include \char`\"{}../include/structure.\+h\char`\"{}}\\*
{\ttfamily \#include \char`\"{}../include/couleur.\+h\char`\"{}}\\*
\subsection*{Macros}
\begin{DoxyCompactItemize}
\item 
\hypertarget{affichage_8c_a0240ac851181b84ac374872dc5434ee4}{}\#define {\bfseries N}~4\label{affichage_8c_a0240ac851181b84ac374872dc5434ee4}

\end{DoxyCompactItemize}
\subsection*{Fonctions}
\begin{DoxyCompactItemize}
\item 
void \hyperlink{affichage_8c_a1d115f06ae228b9b9d49509db742df3b}{afficher\+\_\+matrice} (\hyperlink{structt__case}{t\+\_\+case} grille\mbox{[}N\mbox{]}\mbox{[}N\mbox{]})
\begin{DoxyCompactList}\small\item\em Fonction qui permet d\textquotesingle{}afficher la grille et sa légende. \end{DoxyCompactList}\item 
void \hyperlink{affichage_8c_a8c8c4131e09e8a071633cd0aaa816c20}{afficher\+\_\+liste} (int nbmot, \hyperlink{structt__score}{t\+\_\+score} T\mbox{[}1000\mbox{]})
\begin{DoxyCompactList}\small\item\em Fonction qui permet d\textquotesingle{}afficher la liste des mots trié en fonction de leur score. \end{DoxyCompactList}\end{DoxyCompactItemize}


\subsection{Description détaillée}
Fichier qui contient les fonctions qui permettent de réaliser de l\textquotesingle{}affichage. 

\begin{DoxyAuthor}{Auteur}
B\+O\+U\+V\+E\+T Rémi \& P\+R\+A\+D\+E\+R\+E-\/\+N\+I\+Q\+U\+E\+T Alexandre 
\end{DoxyAuthor}
\begin{DoxyVersion}{Version}
1.\+0 
\end{DoxyVersion}
\begin{DoxyDate}{Date}
15 novembre 2015 
\end{DoxyDate}


\subsection{Documentation des fonctions}
\hypertarget{affichage_8c_a8c8c4131e09e8a071633cd0aaa816c20}{}\index{affichage.\+c@{affichage.\+c}!afficher\+\_\+liste@{afficher\+\_\+liste}}
\index{afficher\+\_\+liste@{afficher\+\_\+liste}!affichage.\+c@{affichage.\+c}}
\subsubsection[{afficher\+\_\+liste(int nbmot, t\+\_\+score T[1000])}]{\setlength{\rightskip}{0pt plus 5cm}void afficher\+\_\+liste (
\begin{DoxyParamCaption}
\item[{int}]{compteur, }
\item[{{\bf t\+\_\+score}}]{T\mbox{[}1000\mbox{]}}
\end{DoxyParamCaption}
)}\label{affichage_8c_a8c8c4131e09e8a071633cd0aaa816c20}


Fonction qui permet d\textquotesingle{}afficher la liste des mots trié en fonction de leur score. 


\begin{DoxyParams}{Paramètres}
{\em Prend} & en paramètre le nombre de mot et le tableau des mots et des scores. \\
\hline
\end{DoxyParams}
\begin{DoxyReturn}{Renvoie}
Ne retourne rien. 
\end{DoxyReturn}


Définition à la ligne 75 du fichier affichage.\+c.

\hypertarget{affichage_8c_a1d115f06ae228b9b9d49509db742df3b}{}\index{affichage.\+c@{affichage.\+c}!afficher\+\_\+matrice@{afficher\+\_\+matrice}}
\index{afficher\+\_\+matrice@{afficher\+\_\+matrice}!affichage.\+c@{affichage.\+c}}
\subsubsection[{afficher\+\_\+matrice(t\+\_\+case grille[N][N])}]{\setlength{\rightskip}{0pt plus 5cm}void afficher\+\_\+matrice (
\begin{DoxyParamCaption}
\item[{{\bf t\+\_\+case}}]{grille\mbox{[}\+N\mbox{]}\mbox{[}\+N\mbox{]}}
\end{DoxyParamCaption}
)}\label{affichage_8c_a1d115f06ae228b9b9d49509db742df3b}


Fonction qui permet d\textquotesingle{}afficher la grille et sa légende. 


\begin{DoxyParams}{Paramètres}
{\em Prend} & en paramètre la grille. \\
\hline
\end{DoxyParams}
\begin{DoxyReturn}{Renvoie}
Ne retourne rien. 
\end{DoxyReturn}


Définition à la ligne 20 du fichier affichage.\+c.


\hypertarget{generer_8c}{}\section{Référence du fichier src/generer.c}
\label{generer_8c}\index{src/generer.\+c@{src/generer.\+c}}


Fichier qui contient les fonctions qui permettent generer la grille.  


{\ttfamily \#include $<$stdlib.\+h$>$}\\*
{\ttfamily \#include $<$time.\+h$>$}\\*
{\ttfamily \#include \char`\"{}../include/structure.\+h\char`\"{}}\\*
\subsection*{Macros}
\begin{DoxyCompactItemize}
\item 
\hypertarget{generer_8c_a0240ac851181b84ac374872dc5434ee4}{}\#define {\bfseries N}~4\label{generer_8c_a0240ac851181b84ac374872dc5434ee4}

\end{DoxyCompactItemize}
\subsection*{Fonctions}
\begin{DoxyCompactItemize}
\item 
int \hyperlink{generer_8c_a31044302280c0ec9eca89286f81127af}{rand\+\_\+a\+\_\+b} (int a, int b)
\begin{DoxyCompactList}\small\item\em Fonction qui permet de générer un nombre aléatoire. \end{DoxyCompactList}\item 
void \hyperlink{generer_8c_a7ce74e6424ae2e628d8fea22f43beeb2}{generation} (\hyperlink{structt__case}{t\+\_\+case} grille\mbox{[}N\mbox{]}\mbox{[}N\mbox{]})
\begin{DoxyCompactList}\small\item\em Fonction qui permet de générer la grille de caractère ainsi que les bonus qui lui son associé. \end{DoxyCompactList}\end{DoxyCompactItemize}


\subsection{Description détaillée}
Fichier qui contient les fonctions qui permettent generer la grille. 

\begin{DoxyAuthor}{Auteur}
B\+O\+U\+V\+E\+T Rémi \& P\+R\+A\+D\+E\+R\+E-\/\+N\+I\+Q\+U\+E\+T Alexandre 
\end{DoxyAuthor}
\begin{DoxyVersion}{Version}
1.\+0 
\end{DoxyVersion}
\begin{DoxyDate}{Date}
15 novembre 2015 
\end{DoxyDate}


\subsection{Documentation des fonctions}
\hypertarget{generer_8c_a7ce74e6424ae2e628d8fea22f43beeb2}{}\index{generer.\+c@{generer.\+c}!generation@{generation}}
\index{generation@{generation}!generer.\+c@{generer.\+c}}
\subsubsection[{generation(t\+\_\+case grille[N][N])}]{\setlength{\rightskip}{0pt plus 5cm}void generation (
\begin{DoxyParamCaption}
\item[{{\bf t\+\_\+case}}]{grille\mbox{[}\+N\mbox{]}\mbox{[}\+N\mbox{]}}
\end{DoxyParamCaption}
)}\label{generer_8c_a7ce74e6424ae2e628d8fea22f43beeb2}


Fonction qui permet de générer la grille de caractère ainsi que les bonus qui lui son associé. 


\begin{DoxyParams}{Paramètres}
{\em Prend} & en paramètre la grille. \\
\hline
\end{DoxyParams}
\begin{DoxyReturn}{Renvoie}
Ne retourne rien. 
\end{DoxyReturn}


Définition à la ligne 33 du fichier generer.\+c.

\hypertarget{generer_8c_a31044302280c0ec9eca89286f81127af}{}\index{generer.\+c@{generer.\+c}!rand\+\_\+a\+\_\+b@{rand\+\_\+a\+\_\+b}}
\index{rand\+\_\+a\+\_\+b@{rand\+\_\+a\+\_\+b}!generer.\+c@{generer.\+c}}
\subsubsection[{rand\+\_\+a\+\_\+b(int a, int b)}]{\setlength{\rightskip}{0pt plus 5cm}int rand\+\_\+a\+\_\+b (
\begin{DoxyParamCaption}
\item[{int}]{a, }
\item[{int}]{b}
\end{DoxyParamCaption}
)}\label{generer_8c_a31044302280c0ec9eca89286f81127af}


Fonction qui permet de générer un nombre aléatoire. 


\begin{DoxyParams}{Paramètres}
{\em Prend} & en paramètre deux entier qui permettent de définir les bornes du nombre aléatoire généré avec \mbox{[}a;b\mbox{[} . \\
\hline
\end{DoxyParams}
\begin{DoxyReturn}{Renvoie}
Retourne le nombre aléatoire. 
\end{DoxyReturn}


Définition à la ligne 22 du fichier generer.\+c.


\hypertarget{main_8c}{}\section{Référence du fichier src/main.c}
\label{main_8c}\index{src/main.\+c@{src/main.\+c}}


Fichier qui contient la fonction main et donc les appels des fonctions principales.  


{\ttfamily \#include $<$stdlib.\+h$>$}\\*
{\ttfamily \#include $<$stdio.\+h$>$}\\*
{\ttfamily \#include $<$unistd.\+h$>$}\\*
{\ttfamily \#include $<$time.\+h$>$}\\*
{\ttfamily \#include $<$string.\+h$>$}\\*
{\ttfamily \#include \char`\"{}../include/pile\+\_\+ptr.\+h\char`\"{}}\\*
{\ttfamily \#include \char`\"{}../include/structure.\+h\char`\"{}}\\*
{\ttfamily \#include \char`\"{}../include/affichage.\+h\char`\"{}}\\*
{\ttfamily \#include \char`\"{}../include/points.\+h\char`\"{}}\\*
{\ttfamily \#include \char`\"{}../include/generer.\+h\char`\"{}}\\*
{\ttfamily \#include \char`\"{}../include/traitement.\+h\char`\"{}}\\*
\subsection*{Macros}
\begin{DoxyCompactItemize}
\item 
\hypertarget{main_8c_a0240ac851181b84ac374872dc5434ee4}{}\#define {\bfseries N}~4\label{main_8c_a0240ac851181b84ac374872dc5434ee4}

\end{DoxyCompactItemize}
\subsection*{Fonctions}
\begin{DoxyCompactItemize}
\item 
int \hyperlink{main_8c_a840291bc02cba5474a4cb46a9b9566fe}{main} (void)
\begin{DoxyCompactList}\small\item\em Fonction main. \end{DoxyCompactList}\end{DoxyCompactItemize}


\subsection{Description détaillée}
Fichier qui contient la fonction main et donc les appels des fonctions principales. 

\begin{DoxyAuthor}{Auteur}
B\+O\+U\+V\+E\+T Rémi \& P\+R\+A\+D\+E\+R\+E-\/\+N\+I\+Q\+U\+E\+T Alexandre 
\end{DoxyAuthor}
\begin{DoxyVersion}{Version}
1.\+0 
\end{DoxyVersion}
\begin{DoxyDate}{Date}
15 novembre 2015 
\end{DoxyDate}


\subsection{Documentation des fonctions}
\hypertarget{main_8c_a840291bc02cba5474a4cb46a9b9566fe}{}\index{main.\+c@{main.\+c}!main@{main}}
\index{main@{main}!main.\+c@{main.\+c}}
\subsubsection[{main(void)}]{\setlength{\rightskip}{0pt plus 5cm}int main (
\begin{DoxyParamCaption}
\item[{void}]{}
\end{DoxyParamCaption}
)}\label{main_8c_a840291bc02cba5474a4cb46a9b9566fe}


Fonction main. 


\begin{DoxyParams}{Paramètres}
{\em Ne} & prend pas de paramètres. \\
\hline
\end{DoxyParams}
\begin{DoxyReturn}{Renvoie}
Retourne 0; 
\end{DoxyReturn}


Définition à la ligne 30 du fichier main.\+c.


\hypertarget{pile__ptr_8c}{}\section{Référence du fichier src/pile\+\_\+ptr.c}
\label{pile__ptr_8c}\index{src/pile\+\_\+ptr.\+c@{src/pile\+\_\+ptr.\+c}}


Fichier qui contient les primitives d\textquotesingle{}accès à la pile.  


{\ttfamily \#include $<$stdlib.\+h$>$}\\*
{\ttfamily \#include $<$string.\+h$>$}\\*
{\ttfamily \#include \char`\"{}../include/structure.\+h\char`\"{}}\\*
\subsection*{Macros}
\begin{DoxyCompactItemize}
\item 
\hypertarget{pile__ptr_8c_a0240ac851181b84ac374872dc5434ee4}{}\#define {\bfseries N}~4\label{pile__ptr_8c_a0240ac851181b84ac374872dc5434ee4}

\end{DoxyCompactItemize}
\subsection*{Fonctions}
\begin{DoxyCompactItemize}
\item 
void \hyperlink{pile__ptr_8c_ac4e8451a9141ddd60fd45e98cf741aad}{initpile} (void)
\begin{DoxyCompactList}\small\item\em Fonction qui initialise la pile. \end{DoxyCompactList}\item 
void \hyperlink{pile__ptr_8c_a7024e12f28fa12e7835b8940054ed4fa}{Copier\+Matrice} (char T1\mbox{[}N\mbox{]}\mbox{[}N\mbox{]}, char T2\mbox{[}N\mbox{]}\mbox{[}N\mbox{]})
\begin{DoxyCompactList}\small\item\em Fonction qui copie une matrice de caractère dans une autre. \end{DoxyCompactList}\item 
void \hyperlink{pile__ptr_8c_afd2bb36dec2c145c8f3a581ba51819ed}{empiler} (int x2, int y2, char chemin2\mbox{[}N\mbox{]}\mbox{[}N\mbox{]})
\begin{DoxyCompactList}\small\item\em Fonction qui ajoute un élément à notre pile. \end{DoxyCompactList}\item 
void \hyperlink{pile__ptr_8c_a2e632b78b736358faa9b599fd08e54e5}{depiler} (int $\ast$x2, int $\ast$y2, char chemin2\mbox{[}N\mbox{]}\mbox{[}N\mbox{]})
\begin{DoxyCompactList}\small\item\em Fonction qui permet de retirer un élément de la pile et d\textquotesingle{}attribuer sa valeur aux élément placés en paramètre. \end{DoxyCompactList}\end{DoxyCompactItemize}
\subsection*{Variables}
\begin{DoxyCompactItemize}
\item 
\hypertarget{pile__ptr_8c_ae36f32fca1dd25d65ba228c41d26b40f}{}\hyperlink{structelement}{t\+\_\+element} $\ast$ {\bfseries pile}\label{pile__ptr_8c_ae36f32fca1dd25d65ba228c41d26b40f}

\end{DoxyCompactItemize}


\subsection{Description détaillée}
Fichier qui contient les primitives d\textquotesingle{}accès à la pile. 

\begin{DoxyAuthor}{Auteur}
B\+O\+U\+V\+E\+T Rémi \& P\+R\+A\+D\+E\+R\+E-\/\+N\+I\+Q\+U\+E\+T Alexandre 
\end{DoxyAuthor}
\begin{DoxyVersion}{Version}
1.\+0 
\end{DoxyVersion}
\begin{DoxyDate}{Date}
15 novembre 2015 
\end{DoxyDate}


\subsection{Documentation des fonctions}
\hypertarget{pile__ptr_8c_a7024e12f28fa12e7835b8940054ed4fa}{}\index{pile\+\_\+ptr.\+c@{pile\+\_\+ptr.\+c}!Copier\+Matrice@{Copier\+Matrice}}
\index{Copier\+Matrice@{Copier\+Matrice}!pile\+\_\+ptr.\+c@{pile\+\_\+ptr.\+c}}
\subsubsection[{Copier\+Matrice(char T1[N][N], char T2[N][N])}]{\setlength{\rightskip}{0pt plus 5cm}void Copier\+Matrice (
\begin{DoxyParamCaption}
\item[{char}]{T1\mbox{[}\+N\mbox{]}\mbox{[}\+N\mbox{]}, }
\item[{char}]{T2\mbox{[}\+N\mbox{]}\mbox{[}\+N\mbox{]}}
\end{DoxyParamCaption}
)}\label{pile__ptr_8c_a7024e12f28fa12e7835b8940054ed4fa}


Fonction qui copie une matrice de caractère dans une autre. 


\begin{DoxyParams}{Paramètres}
{\em Il} & y a deux mattrice de caractère en paramètre \+: T2 qui se copie dans T1. \\
\hline
\end{DoxyParams}
\begin{DoxyReturn}{Renvoie}
Ne retourne rien. 
\end{DoxyReturn}


Définition à la ligne 37 du fichier pile\+\_\+ptr.\+c.

\hypertarget{pile__ptr_8c_a2e632b78b736358faa9b599fd08e54e5}{}\index{pile\+\_\+ptr.\+c@{pile\+\_\+ptr.\+c}!depiler@{depiler}}
\index{depiler@{depiler}!pile\+\_\+ptr.\+c@{pile\+\_\+ptr.\+c}}
\subsubsection[{depiler(int $\ast$x2, int $\ast$y2, char chemin2[N][N])}]{\setlength{\rightskip}{0pt plus 5cm}void depiler (
\begin{DoxyParamCaption}
\item[{int $\ast$}]{x2, }
\item[{int $\ast$}]{y2, }
\item[{char}]{chemin2\mbox{[}\+N\mbox{]}\mbox{[}\+N\mbox{]}}
\end{DoxyParamCaption}
)}\label{pile__ptr_8c_a2e632b78b736358faa9b599fd08e54e5}


Fonction qui permet de retirer un élément de la pile et d\textquotesingle{}attribuer sa valeur aux élément placés en paramètre. 


\begin{DoxyParams}{Paramètres}
{\em Elle} & prend en paramètre les pointeurs $\ast$x2, $\ast$y2 et chemin2. \\
\hline
\end{DoxyParams}
\begin{DoxyReturn}{Renvoie}
Ne retourne rien. 
\end{DoxyReturn}


Définition à la ligne 72 du fichier pile\+\_\+ptr.\+c.

\hypertarget{pile__ptr_8c_afd2bb36dec2c145c8f3a581ba51819ed}{}\index{pile\+\_\+ptr.\+c@{pile\+\_\+ptr.\+c}!empiler@{empiler}}
\index{empiler@{empiler}!pile\+\_\+ptr.\+c@{pile\+\_\+ptr.\+c}}
\subsubsection[{empiler(int x2, int y2, char chemin2[N][N])}]{\setlength{\rightskip}{0pt plus 5cm}void empiler (
\begin{DoxyParamCaption}
\item[{int}]{x2, }
\item[{int}]{y2, }
\item[{char}]{chemin2\mbox{[}\+N\mbox{]}\mbox{[}\+N\mbox{]}}
\end{DoxyParamCaption}
)}\label{pile__ptr_8c_afd2bb36dec2c145c8f3a581ba51819ed}


Fonction qui ajoute un élément à notre pile. 


\begin{DoxyParams}{Paramètres}
{\em Prend} & en paramètre deux coordonnées x2 et y2 et également la matrice de caractère chemin2. \\
\hline
\end{DoxyParams}
\begin{DoxyReturn}{Renvoie}
Ne retourne rien. 
\end{DoxyReturn}


Définition à la ligne 54 du fichier pile\+\_\+ptr.\+c.

\hypertarget{pile__ptr_8c_ac4e8451a9141ddd60fd45e98cf741aad}{}\index{pile\+\_\+ptr.\+c@{pile\+\_\+ptr.\+c}!initpile@{initpile}}
\index{initpile@{initpile}!pile\+\_\+ptr.\+c@{pile\+\_\+ptr.\+c}}
\subsubsection[{initpile(void)}]{\setlength{\rightskip}{0pt plus 5cm}void initpile (
\begin{DoxyParamCaption}
\item[{void}]{}
\end{DoxyParamCaption}
)}\label{pile__ptr_8c_ac4e8451a9141ddd60fd45e98cf741aad}


Fonction qui initialise la pile. 


\begin{DoxyParams}{Paramètres}
{\em Aucun} & paramètre. \\
\hline
\end{DoxyParams}
\begin{DoxyReturn}{Renvoie}
Ne retourne rien. 
\end{DoxyReturn}


Définition à la ligne 26 du fichier pile\+\_\+ptr.\+c.


\hypertarget{points_8c}{}\section{Référence du fichier src/points.c}
\label{points_8c}\index{src/points.\+c@{src/points.\+c}}


Fichier qui contient toutes les fonctions qui comptent le nombre de point d\textquotesingle{}un mot.  


{\ttfamily \#include $<$stdio.\+h$>$}\\*
{\ttfamily \#include $<$string.\+h$>$}\\*
{\ttfamily \#include \char`\"{}../include/structure.\+h\char`\"{}}\\*
\subsection*{Macros}
\begin{DoxyCompactItemize}
\item 
\hypertarget{points_8c_a0240ac851181b84ac374872dc5434ee4}{}\#define {\bfseries N}~4\label{points_8c_a0240ac851181b84ac374872dc5434ee4}

\end{DoxyCompactItemize}
\subsection*{Fonctions}
\begin{DoxyCompactItemize}
\item 
int \hyperlink{points_8c_a09c563917508ec5f7b30f5e51a2f41bd}{points\+\_\+lettre} (char lettre)
\begin{DoxyCompactList}\small\item\em Fonction qui permet de connaitre le nombre de point initial d\textquotesingle{}une lettre. \end{DoxyCompactList}\item 
int \hyperlink{points_8c_aa91baeedbdca68768f85f12d05a73214}{points\+\_\+lettre\+\_\+bonus} (\hyperlink{structt__case}{t\+\_\+case} grille\mbox{[}N\mbox{]}\mbox{[}N\mbox{]}, char chemin\mbox{[}N\mbox{]}\mbox{[}N\mbox{]})
\begin{DoxyCompactList}\small\item\em Fonction qui permet de calculer le nombre de point de chaque lettre et de leur bonus associé. \end{DoxyCompactList}\item 
int \hyperlink{points_8c_a26cbb9e7dd284b9268a40830c80cd1bf}{points\+\_\+longueur} (char mot\mbox{[}40\mbox{]})
\begin{DoxyCompactList}\small\item\em Fonction qui permet de calculer nombre de points associé à la longueur du mot. \end{DoxyCompactList}\item 
void \hyperlink{points_8c_af7b76ac111b0aec7744926a43b9f3502}{points\+\_\+mot\+\_\+bonus} (int $\ast$nb\+Points, \hyperlink{structt__case}{t\+\_\+case} grille\mbox{[}N\mbox{]}\mbox{[}N\mbox{]}, char chemin\mbox{[}N\mbox{]}\mbox{[}N\mbox{]})
\begin{DoxyCompactList}\small\item\em Fonction qui permet de calculer le bonus au niveau du mot. \end{DoxyCompactList}\item 
int \hyperlink{points_8c_a2eb22ab537e59f75c27ce523ae3a52e0}{calcul\+Point} (\hyperlink{structt__case}{t\+\_\+case} grille\mbox{[}N\mbox{]}\mbox{[}N\mbox{]}, char mot\mbox{[}40\mbox{]}, char chemin\mbox{[}N\mbox{]}\mbox{[}N\mbox{]})
\begin{DoxyCompactList}\small\item\em Fonction qui permet de calculer le nombre total de point d\textquotesingle{}un mot en prenant en compte sa longueur et ses bonus. \end{DoxyCompactList}\end{DoxyCompactItemize}


\subsection{Description détaillée}
Fichier qui contient toutes les fonctions qui comptent le nombre de point d\textquotesingle{}un mot. 

\begin{DoxyAuthor}{Auteur}
B\+O\+U\+V\+E\+T Rémi \& P\+R\+A\+D\+E\+R\+E-\/\+N\+I\+Q\+U\+E\+T Alexandre 
\end{DoxyAuthor}
\begin{DoxyVersion}{Version}
1.\+0 
\end{DoxyVersion}
\begin{DoxyDate}{Date}
15 novembre 2015 
\end{DoxyDate}


\subsection{Documentation des fonctions}
\hypertarget{points_8c_a2eb22ab537e59f75c27ce523ae3a52e0}{}\index{points.\+c@{points.\+c}!calcul\+Point@{calcul\+Point}}
\index{calcul\+Point@{calcul\+Point}!points.\+c@{points.\+c}}
\subsubsection[{calcul\+Point(t\+\_\+case grille[N][N], char mot[40], char chemin[N][N])}]{\setlength{\rightskip}{0pt plus 5cm}int calcul\+Point (
\begin{DoxyParamCaption}
\item[{{\bf t\+\_\+case}}]{grille\mbox{[}\+N\mbox{]}\mbox{[}\+N\mbox{]}, }
\item[{char}]{motdico\mbox{[}40\mbox{]}, }
\item[{char}]{chemin\mbox{[}\+N\mbox{]}\mbox{[}\+N\mbox{]}}
\end{DoxyParamCaption}
)}\label{points_8c_a2eb22ab537e59f75c27ce523ae3a52e0}


Fonction qui permet de calculer le nombre total de point d\textquotesingle{}un mot en prenant en compte sa longueur et ses bonus. 


\begin{DoxyParams}{Paramètres}
{\em Prend} & en paramètre la grille du jeu, le mot présent dans la grille et son chemin associé. \\
\hline
\end{DoxyParams}
\begin{DoxyReturn}{Renvoie}
Retourne le nombre de point associé au mot. 
\end{DoxyReturn}


Définition à la ligne 133 du fichier points.\+c.

\hypertarget{points_8c_a09c563917508ec5f7b30f5e51a2f41bd}{}\index{points.\+c@{points.\+c}!points\+\_\+lettre@{points\+\_\+lettre}}
\index{points\+\_\+lettre@{points\+\_\+lettre}!points.\+c@{points.\+c}}
\subsubsection[{points\+\_\+lettre(char lettre)}]{\setlength{\rightskip}{0pt plus 5cm}int points\+\_\+lettre (
\begin{DoxyParamCaption}
\item[{char}]{lettre}
\end{DoxyParamCaption}
)}\label{points_8c_a09c563917508ec5f7b30f5e51a2f41bd}


Fonction qui permet de connaitre le nombre de point initial d\textquotesingle{}une lettre. 


\begin{DoxyParams}{Paramètres}
{\em Prend} & un caractère en paramètre. \\
\hline
\end{DoxyParams}
\begin{DoxyReturn}{Renvoie}
Retourne le nombre de point associé à la lettre. 
\end{DoxyReturn}


Définition à la ligne 21 du fichier points.\+c.

\hypertarget{points_8c_aa91baeedbdca68768f85f12d05a73214}{}\index{points.\+c@{points.\+c}!points\+\_\+lettre\+\_\+bonus@{points\+\_\+lettre\+\_\+bonus}}
\index{points\+\_\+lettre\+\_\+bonus@{points\+\_\+lettre\+\_\+bonus}!points.\+c@{points.\+c}}
\subsubsection[{points\+\_\+lettre\+\_\+bonus(t\+\_\+case grille[N][N], char chemin[N][N])}]{\setlength{\rightskip}{0pt plus 5cm}int points\+\_\+lettre\+\_\+bonus (
\begin{DoxyParamCaption}
\item[{{\bf t\+\_\+case}}]{grille\mbox{[}\+N\mbox{]}\mbox{[}\+N\mbox{]}, }
\item[{char}]{chemin\mbox{[}\+N\mbox{]}\mbox{[}\+N\mbox{]}}
\end{DoxyParamCaption}
)}\label{points_8c_aa91baeedbdca68768f85f12d05a73214}


Fonction qui permet de calculer le nombre de point de chaque lettre et de leur bonus associé. 


\begin{DoxyParams}{Paramètres}
{\em Prend} & en paramètre la grille et le chemin associé. \\
\hline
\end{DoxyParams}
\begin{DoxyReturn}{Renvoie}
Retourne le nombre de point associé aux lettres du mot. 
\end{DoxyReturn}


Définition à la ligne 59 du fichier points.\+c.

\hypertarget{points_8c_a26cbb9e7dd284b9268a40830c80cd1bf}{}\index{points.\+c@{points.\+c}!points\+\_\+longueur@{points\+\_\+longueur}}
\index{points\+\_\+longueur@{points\+\_\+longueur}!points.\+c@{points.\+c}}
\subsubsection[{points\+\_\+longueur(char mot[40])}]{\setlength{\rightskip}{0pt plus 5cm}int points\+\_\+longueur (
\begin{DoxyParamCaption}
\item[{char}]{mot\mbox{[}40\mbox{]}}
\end{DoxyParamCaption}
)}\label{points_8c_a26cbb9e7dd284b9268a40830c80cd1bf}


Fonction qui permet de calculer nombre de points associé à la longueur du mot. 


\begin{DoxyParams}{Paramètres}
{\em Prend} & en paramètre un mot. \\
\hline
\end{DoxyParams}
\begin{DoxyReturn}{Renvoie}
Retourne le nombre de point de la longueur du mot. 
\end{DoxyReturn}


Définition à la ligne 85 du fichier points.\+c.

\hypertarget{points_8c_af7b76ac111b0aec7744926a43b9f3502}{}\index{points.\+c@{points.\+c}!points\+\_\+mot\+\_\+bonus@{points\+\_\+mot\+\_\+bonus}}
\index{points\+\_\+mot\+\_\+bonus@{points\+\_\+mot\+\_\+bonus}!points.\+c@{points.\+c}}
\subsubsection[{points\+\_\+mot\+\_\+bonus(int $\ast$nb\+Points, t\+\_\+case grille[N][N], char chemin[N][N])}]{\setlength{\rightskip}{0pt plus 5cm}void points\+\_\+mot\+\_\+bonus (
\begin{DoxyParamCaption}
\item[{int $\ast$}]{nb\+Points, }
\item[{{\bf t\+\_\+case}}]{grille\mbox{[}\+N\mbox{]}\mbox{[}\+N\mbox{]}, }
\item[{char}]{chemin\mbox{[}\+N\mbox{]}\mbox{[}\+N\mbox{]}}
\end{DoxyParamCaption}
)}\label{points_8c_af7b76ac111b0aec7744926a43b9f3502}


Fonction qui permet de calculer le bonus au niveau du mot. 


\begin{DoxyParams}{Paramètres}
{\em Prend} & en paramètre la grille du jeu, le chemin associé et un pointeur sur une varible points. \\
\hline
\end{DoxyParams}
\begin{DoxyReturn}{Renvoie}
Ne retourne rien. 
\end{DoxyReturn}


Définition à la ligne 113 du fichier points.\+c.


\hypertarget{traitement_8c}{\section{Référence du fichier src/traitement.c}
\label{traitement_8c}\index{src/traitement.\-c@{src/traitement.\-c}}
}


Fichier qui contient toutes les fonctions qui traitent la grille.  


{\ttfamily \#include $<$stdio.\-h$>$}\\*
{\ttfamily \#include $<$string.\-h$>$}\\*
{\ttfamily \#include \char`\"{}../include/structure.\-h\char`\"{}}\\*
{\ttfamily \#include \char`\"{}../include/pile\-\_\-ptr.\-h\char`\"{}}\\*
{\ttfamily \#include \char`\"{}../include/points.\-h\char`\"{}}\\*
\subsection*{Macros}
\begin{DoxyCompactItemize}
\item 
\hypertarget{traitement_8c_a0240ac851181b84ac374872dc5434ee4}{\#define {\bfseries N}~4}\label{traitement_8c_a0240ac851181b84ac374872dc5434ee4}

\end{DoxyCompactItemize}
\subsection*{Fonctions}
\begin{DoxyCompactItemize}
\item 
\hypertarget{traitement_8c_ade51c5df2029d0158ed54cde87dfdd25}{int {\bfseries trouverchemin} (\hyperlink{structt__case}{t\-\_\-case} grille\mbox{[}N\mbox{]}\mbox{[}N\mbox{]}, char motdico\mbox{[}40\mbox{]}, int i, int j, int chemin\mbox{[}N\mbox{]}\mbox{[}N\mbox{]}, char chemin\-Mot\mbox{[}N\mbox{]}\mbox{[}N\mbox{]})}\label{traitement_8c_ade51c5df2029d0158ed54cde87dfdd25}

\item 
\hypertarget{traitement_8c_af6697a37ff4d26a73fbe7f90e2159a95}{int {\bfseries motpresent} (\hyperlink{structt__case}{t\-\_\-case} grille\mbox{[}N\mbox{]}\mbox{[}N\mbox{]}, char motdico\mbox{[}40\mbox{]}, int chemin\mbox{[}N\mbox{]}\mbox{[}N\mbox{]}, char chemin\-Mot\mbox{[}N\mbox{]}\mbox{[}N\mbox{]})}\label{traitement_8c_af6697a37ff4d26a73fbe7f90e2159a95}

\item 
void \hyperlink{traitement_8c_af38304f3718d12ae5f3424a12a841103}{trouver\-Liste} (\hyperlink{structt__case}{t\-\_\-case} grille\mbox{[}N\mbox{]}\mbox{[}N\mbox{]})
\begin{DoxyCompactList}\small\item\em Fonction qui enregistre dans un fichier temporaire tous les mots présents dans la matrice ainsi que leurs points associés. \end{DoxyCompactList}\item 
void \hyperlink{traitement_8c_ab160ae699802dc5e10cda59e2d335e73}{tri} (\hyperlink{structt__score}{t\-\_\-score} T\mbox{[}1000\mbox{]}, int $\ast$taille\-Liste)
\begin{DoxyCompactList}\small\item\em Fonction qui récupère tous les mots et leurs points correspondant pour les trier. \end{DoxyCompactList}\end{DoxyCompactItemize}


\subsection{Description détaillée}
Fichier qui contient toutes les fonctions qui traitent la grille. \begin{DoxyAuthor}{Auteur}
B\-O\-U\-V\-E\-T Rémi \& P\-R\-A\-D\-E\-R\-E-\/\-N\-I\-Q\-U\-E\-T Alexandre 
\end{DoxyAuthor}
\begin{DoxyVersion}{Version}
1.\-0 
\end{DoxyVersion}
\begin{DoxyDate}{Date}
15 novembre 2015 
\end{DoxyDate}


Définition dans le fichier \hyperlink{traitement_8c_source}{traitement.\-c}.



\subsection{Documentation des fonctions}
\hypertarget{traitement_8c_ab160ae699802dc5e10cda59e2d335e73}{\index{traitement.\-c@{traitement.\-c}!tri@{tri}}
\index{tri@{tri}!traitement.c@{traitement.\-c}}
\subsubsection[{tri}]{\setlength{\rightskip}{0pt plus 5cm}void tri (
\begin{DoxyParamCaption}
\item[{{\bf t\-\_\-score}}]{T\mbox{[}1000\mbox{]}, }
\item[{int $\ast$}]{taille\-Liste}
\end{DoxyParamCaption}
)}}\label{traitement_8c_ab160ae699802dc5e10cda59e2d335e73}


Fonction qui récupère tous les mots et leurs points correspondant pour les trier. 

void \hyperlink{traitement_8h_ab160ae699802dc5e10cda59e2d335e73}{tri(t\-\_\-score T\mbox{[}1000\mbox{]}, int $\ast$ taille\-Liste)} 
\begin{DoxyParams}{Paramètres}
{\em Prend} & en paramètre le tableau qui stocke les mots et leurs points associés ainsi que un pointeur sur le nombre de mot. \\
\hline
\end{DoxyParams}
\begin{DoxyReturn}{Renvoie}
Ne retourne rien. 
\end{DoxyReturn}


Définition à la ligne 218 du fichier traitement.\-c.

\hypertarget{traitement_8c_af38304f3718d12ae5f3424a12a841103}{\index{traitement.\-c@{traitement.\-c}!trouver\-Liste@{trouver\-Liste}}
\index{trouver\-Liste@{trouver\-Liste}!traitement.c@{traitement.\-c}}
\subsubsection[{trouver\-Liste}]{\setlength{\rightskip}{0pt plus 5cm}void trouver\-Liste (
\begin{DoxyParamCaption}
\item[{{\bf t\-\_\-case}}]{grille\mbox{[}\-N\mbox{]}\mbox{[}\-N\mbox{]}}
\end{DoxyParamCaption}
)}}\label{traitement_8c_af38304f3718d12ae5f3424a12a841103}


Fonction qui enregistre dans un fichier temporaire tous les mots présents dans la matrice ainsi que leurs points associés. 


\begin{DoxyParams}{Paramètres}
{\em Prend} & en paramètre la grille. \\
\hline
\end{DoxyParams}
\begin{DoxyReturn}{Renvoie}
Ne retourne rien. 
\end{DoxyReturn}


Définition à la ligne 188 du fichier traitement.\-c.


\addcontentsline{toc}{part}{Index}
\printindex
\end{document}
