Il faut former le plus de mots possibles avec les seize lettres disponibles dans une grille de quatre par quatre. Les mots doivent être au moins de deux lettres, et trouvés en utilisant des lettres adjacentes les unes aux autres sans réutiliser deux fois la même case de la grille. Les formes conjuguées des verbes sont acceptées.

Comme au Scrabble chaque lettre a une valeur en fonction de sa difficulté à être placée dans un mot et certaines cases sont spéciales \-:


\begin{DoxyItemize}
\item Lettre compte double \-: la lettre dans cette case compte deux fois plus de points
\item Lettre compte triple \-: la lettre dans cette case compte trois fois plus de points
\item Mot compte double \-: les mots contenant cette case comptent deux fois plus de points
\item Mot compte triple \-: les mots contenant cette case comptent trois fois plus de points
\end{DoxyItemize}

Il existe également un bonus en fonction de la longueur du mot créé \-:


\begin{DoxyItemize}
\item 5 lettres \-: 5 points supplémentaires
\item 6 lettres \-: 10 points supplémentaires
\item 7 lettres \-: 15 points supplémentaires
\item 8 lettres \-: 20 points supplémentaires
\item 9 lettres ou plus \-: 25 points supplémentaires
\end{DoxyItemize}

\subsection*{Compilation}

\begin{quotation}
make

\end{quotation}


Permet de compiler l'ensemble des sources, l'exécutable généré peut être retrouvé dans ./bin .

\begin{quotation}
make clean

\end{quotation}


Permet de supprimer tous les .o créé lors de la compilation.

\subsection*{Execution}

\begin{quotation}
./bin/ruzzlesolver

\end{quotation}


Permet d'exécuter le programme ruzzlesolver. 